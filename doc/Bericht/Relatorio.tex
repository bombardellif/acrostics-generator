%----------------------------------------------------------------------------------------
%	PACKAGES AND OTHER DOCUMENT CONFIGURATIONS
%----------------------------------------------------------------------------------------

\documentclass{reportAlternative}

%\usepackage[german]{babel}
\usepackage[utf8]{inputenc}
\usepackage{graphicx} % Required to insert images
\usepackage{float}
\usepackage{listings}
\usepackage[nottoc]{tocbibind}

\settocbibname{References}

\graphicspath{ {img/} }


% Margins
\topmargin=-0.45in
\evensidemargin=0in
\oddsidemargin=0in
\textwidth=6.5in
\textheight=9.0in
\headsep=0.25in 

\linespread{1.1} % Line spacing

%----------------------------------------------------------------------------------------
%	TITLE PAGE
%----------------------------------------------------------------------------------------

\title{
\includegraphics[scale=0.5]{tub_logo}\\
\Large{
Technische Universität Berlin\\
Fakultät IV - Fakultät Elektrotechnik und Informatik\\
Fachgebiet Datenbanksysteme und Informationsmanagement
}\\
\vspace{3cm}\textbf{Project Report\\
Generating Acrostics via Paraphrasing and Heuristic Search\\
DBPRO - Database Projects (WS 2014/2015)
}\\
}

\author{
\vspace{2cm}\\
Bruno Fillmann () \\
Fernando Bombardelli da Silva ()\\
Jürgen Bauer () \\
\vspace{2.5cm}William Bombardelli da Silva (wbombardellis@mailbox.tu-berlin.de)}

\date{February 2, 2015} % Insert date here if you want it to appear below your name

%----------------------------------------------------------------------------------------

\begin{document}

\maketitle

%----------------------------------------------------------------------------------------
%	TABLE OF CONTENTS
%----------------------------------------------------------------------------------------

%\setcounter{tocdepth}{1} % Uncomment this line if you don't want subsections listed in the ToC

%\newpage
\tableofcontents
\newpage

%----------------------------------------------------------------------------------------
% Examples of elements for the report in latex
%----------------------------------------------------------------------------------------
%================
%Inserting code:
%\lstinputlisting[frame=single, breaklines=true, language=java, label=lst:Foo, caption=Foo]{foo.java}
%================
%Inserting code inline:
%\begin{lstlisting}[frame=single, breaklines=true, language=java]
%//CODE HERE
%\end{lstlisting}
%================
%Define a label for furutre referenceing: \label{sec:Foo} 
%Refecing a label: \ref{sec:Foo}
%================
%Citing Bibliography: \cite{NameOfBibItem}
%================
%Emphasis on text: \emph{foo}
%Bold on text: \textbf{Foo}
%================
%Table Example: 
%\begin{table}[h]
%\centering
%\begin{tabular}{l | l | l | l | l}
%	\hline
%	\textbf{Modificador} & \textbf{Classe} & \textbf{Pacote} & \textbf{Subclasse} & \textbf{Mundo} \\ \hline
%		\textbf{\textit{public}} &	S &		S &	S &	S \\ \hline
%		\textbf{\textit{protected}} &	S &		S &	S &	N \\ \hline
%		\textit{sem modificador} & S &	S &	N &	N \\ \hline
%		\textbf{\textit{private}} &	S &		N &	N &	N \\
%	\hline
%\end{tabular}
%\caption{Tabela de Modificadores de Acesso de Java}
%\end{table}
%================
%Example of figure
%\begin{figure}[H]
%\centering
%\includegraphics[scale=0.5]{img_1_1}
%\caption{\label{}Imagem -- Diagrama Conceitual de Java}
%\end{figure}
%
%----------------------------------------------------------------------------------------

%----------------------------------------------------------------------------------------
%	Abstract
%----------------------------------------------------------------------------------------
\begin{abstract}
ascasdas
\end{abstract}

%----------------------------------------------------------------------------------------
%	Introduction and Motivation
%----------------------------------------------------------------------------------------
\chapter{Introduction and Motivation}
lnmkm
%----------------------------------------------------------------------------------------
%	Description of own work
%----------------------------------------------------------------------------------------
\chapter{Generating Acrostics via Paraphrasing and Heursitic Search}

\section{Problem Definition}

\section{Modeling as Search Problem}

\section{Cost Measure}

\section{Operators}

%----------------------------------------------------------------------------------------
%	Evaluation of the Results
%----------------------------------------------------------------------------------------
\chapter{Evaluation of the Results}

%----------------------------------------------------------------------------------------
%	Summary of Findings
%----------------------------------------------------------------------------------------
\chapter{Summary of Findings}

%----------------------------------------------------------------------------------------
%	References
%----------------------------------------------------------------------------------------
\begin{thebibliography}{1}
\bibitem{JavaSobre}
	Oracle.
	\emph{Obtenha Informações sobre a Tecnologia Java}.
	Disponível em: $<$http://www.java.com/pt\_BR/about$>$.
	Acesso em: Junho de 2014.
\bibitem{BENEKE}
	BENEKE, T.
	\emph{Java: Explore the Possibilities}.
	Disponível em: $<$http://www.oracle.com/technetwork/articles/java/ma14-java-cover-2177777.html$>$.
	Acesso em: Junho de 2014.
\end{thebibliography}

%----------------------------------------------------------------------------------------
%	Appendix
%----------------------------------------------------------------------------------------
\appendix
\chapter{Appendix}


\end{document}
